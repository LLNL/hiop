\clearpage
\appendix
\section{Appendix}

\subsubsection{Condensed Linear System}

The condensed approach supports sparse NLPs with no equality constraints.
\begin{align}
&&&&\min_{x\in\mathbb{R}^n} & \hspace{0.5cm} f(x) &&&& \label{spobj_condensed}\\
&&[v_l]&&& \hspace{0.5cm} d_l \leq d(x) \leq d_u  &[v_u]&&&\label{spineq_condensed} \\
&&[z_l]&&& \hspace{0.5cm} x_l \leq x \leq x_u & [z_u] &&&\label{spbounds_condensed}
\end{align}
Here $f:\mathbb{R}^n\rightarrow\mathbb{R}$, $d:\mathbb{R}^n\rightarrow\mathbb{R}^{m_I}$. The bounds appearing in the inequality constraints~\eqref{spineq_condensed} are assumed to be $d^l\in\mathbb{R}^{m_I}\cup\{-\infty\}$, $d^u\in\mathbb{R}^{m_I}\cup\{+\infty\}$, $d_i^l < d_i^u$, and at least of one of $d_i^l$ and $d_i^u$ are finite for all $i\in\{1,\ldots,m_I\}$. The bounds in~\eqref{spbounds_condensed} are such that $x^l\in\mathbb{R}^{n}\cup\{-\infty\}$, $x^u\in\mathbb{R}^{n}\cup\{+\infty\}$, and $x_i^l < x_i^u$, $i\in\{1,\ldots,n\}$. The quantities insides brackets are the Lagrange multipliers of the constraints. Whenever a bound is infinite, the corresponding multiplier is by convention zero.
Internally, a slack variable $s$ is introduced and the inequality constraints~\eqref{spineq_condensed} are replaced by additional equality constraints and boundary constraints:
\begin{align}
&&&&& \hspace{0.5cm} d(x) = s &[y_d]&&& \\
&&[v_l]&&& \hspace{0.5cm} d_l \leq s \leq d_u  &[v_u]&&&\label{spineq_s} 
\end{align}

\warningcp{Note:} If equality constraints $c(x)=c_E$ are presented, they will be slightly relaxed to inequalities $c_E - C_1\leq c(x)\leq c_E+C_1$, where $C_1$ is a small positive perturbation that will be updated by \Hi internally. Concequently, \Hi still treats the problem with form \eqref{spobj_condensed}-\eqref{spbounds_condensed}.\\

Internally, condensed linear system sloves the most stable `xdycyd' KKT linear system 
\begin{equation} \label{KKT_xdycyd_condensed}
  \begin{bmatrix} 
  H+D_x+\delta_{w}I & 0  & J_d^T\\
  0  & D_d + \delta_{w}I &  -I\\
  J_d & -I & 0 
  \end{bmatrix}
  \begin{bmatrix} \Delta x \\ \Delta d \\ \Delta y_d  \end{bmatrix} = 
  \begin{bmatrix} r_x \\ r_d \\ r_{y_d}\end{bmatrix} 
\end{equation}
by solving the following sequence of linear systems
\begin{align}
  Q & := H+D_x+\delta_wI + J_d^T(D_d+\delta_w I)J_d   \\
  Q\Delta x & = r_x + J_d^T(D_d+\delta_w I)r_{y_d} + J_d^T r_d \label{KKT_condensed} \\
  \Delta d & = J_d \Delta x- r_{y_d} \\
  \Delta y_d   & = D_d \Delta d - r_d 
\end{align}
Equation \eqref{KKT_condensed} is defined as the condensed linear system, where matrix $Q$ is positive definite, guaranteed by the algorithmic mechanism. Therefore, \Hi is able to solve \eqref{KKT_condensed} by a sparse Cholesky solver. 

Since the development of an efficient and robust Cholesky solver is much more mature than the development of an indefinite linear solver for the `xdycyd' linear system on GPU, this condensed linear system can achieve GPU accelaration more naturally by using vendor's standard libraries, e.g., cuSolver inside the CUDA Tookit.



\subsubsection{Normal Equation}

The normal equation approach supports sparse LP or QP with the form \eqref{spobj}-\eqref{spbounds}, where $f:\mathbb{R}^n\rightarrow\mathbb{R}$ is a linear or quadratic function; $c:\mathbb{R}^n\rightarrow\mathbb{R}^{m_E}$ and $d:\mathbb{R}^n\rightarrow\mathbb{R}^{m_I}$ are linear functions.

\warningcp{Note:} If equality constraints $c(x)=c_E$ are presented, they will be slightly relaxed to inequalities $c_E - C_1\leq c(x)\leq c_E+C_1$, where $C_1$ is a small positive perturbation that will be updated by \Hi internally. Concequently, \Hi still treats the problem with form \eqref{spobj_condensed}-\eqref{spbounds_condensed}.\\

Internally, normal equation sloves the most stable `xdycyd' KKT linear system 
\begin{equation} \label{KKT_xdycyd_normaleqn}
  \begin{bmatrix}
    H+D_x+\delta_{w}I & 0 & J_c^T & J_d^T\\
    0  & D_d + \delta_{w}I &  0    &  -I\\
    J_c & 0 & 0 & 0\\
    J_d & -I & 0 & 0
  \end{bmatrix}
  \begin{bmatrix} \Delta x \\ \Delta d \\ \Delta y_c \\ \Delta y_d  \end{bmatrix} = 
  \begin{bmatrix} r_x \\ r_d \\ r_{y_c} \\ r_{y_d}\end{bmatrix}
\end{equation}
by firstly solving the following condensed linear systems
\begin{equation} \label{KKT_normaleqn}
  K
  \begin{bmatrix} \Delta y_c \\ \Delta y_d  \end{bmatrix} 
  = 
  \begin{bmatrix} \tilde{r}_{y_c} \\ \tilde{r}_{y_d}\end{bmatrix}, \\
\end{equation}
where
\begin{equation} \label{KKT_normaleqn_mat}
  K = 
  \begin{bmatrix}
    J_c & 0 \\
    J_d & -I 
  \end{bmatrix}
  \begin{bmatrix}
    H+D_x+\delta_{w}I & 0 \\
    0  & D_d + \delta_{w}I
  \end{bmatrix}^{-1}
  \begin{bmatrix}
    J_c & 0 \\
    J_d & -I 
  \end{bmatrix}^T
\end{equation}
and
\begin{equation} \label{KKT_normaleqn_rhs}
  \begin{bmatrix} \tilde{r}_{y_c} \\ \tilde{r}_{y_d}\end{bmatrix}
  = 
  \begin{bmatrix}
    J_c & 0 \\
    J_d & -I 
  \end{bmatrix} 
  \begin{bmatrix}
    H+D_x+\delta_{w}I & 0 \\
    0  & D_d + \delta_{w}I
  \end{bmatrix}^{-1}
  \begin{bmatrix} {r}_{x} \\ {r}_{d}\end{bmatrix}
  -
  \begin{bmatrix} r_{y_c} \\ r_{y_d}\end{bmatrix}.
\end{equation}

Since matrix $K$ \eqref{KKT_normaleqn_mat} is forced to be positive definite by the algorithmic mechanism, the normal equaltion system \eqref{KKT_normaleqn} can be solved by a Cholesky solver. Once $\Delta y_c$ and $\Delta y_d$ have been calculated, \Hi uses the following equations to retrieve $\Delta x$ and $\Delta d$: 
\begin{equation} \label{KKT_normaleqn_step_xd}
  \begin{bmatrix} \Delta x \\ \Delta d  \end{bmatrix} 
  = 
  \begin{bmatrix}
    H+D_x+\delta_{w}I & 0 \\
    0  & D_d + \delta_{w}I
  \end{bmatrix}^{-1}
  ( \begin{bmatrix} {r}_{x} \\ {r}_{d}\end{bmatrix}
    -
    \begin{bmatrix}
      J_c^{T} & J_d{T} \\
      0 & -I 
    \end{bmatrix}
    \begin{bmatrix} \Delta y_c \\ \Delta y_d  \end{bmatrix} 
  )
\end{equation}


\ignore{
\subsubsection{Krylov-Subspace Iterative Method}


}


