\subsection{General sparse NLPs requiring \textit{up to second-order} derivative information}
The sparse NLP formulation supports sparse optimization problems and requires Hessians of the objective and constraints in addition to gradients/Jacobian of the objective/constraints.
\begin{align}
&&&&\min_{x\in\mathbb{R}^n} & \hspace{0.5cm} f(x) &&&& \label{spobj}\\
&&&&\textnormal{s.t.} &\hspace{0.5cm}  c(x)=c_E &[y_c]&&&\\
&&[v_l]&&& \hspace{0.5cm} d_l \leq d(x) \leq d_u  &[v_u]&&&\label{spineq} \\
&&[z_u]&&& \hspace{0.5cm} x_l \leq x \leq x_u & [z_u] &&&\label{spbounds}
\end{align}
Here $f:\mathbb{R}^n\rightarrow\mathbb{R}$, $c:\mathbb{R}^n\rightarrow\mathbb{R}^{m_E}$, $d:\mathbb{R}^n\rightarrow\mathbb{R}^{m_I}$. The bounds appearing in the inequality constraints~\eqref{ineq} are assumed to be $d^l\in\mathbb{R}^{m_I}\cup\{-\infty\}$, $d^u\in\mathbb{R}^{m_I}\cup\{+\infty\}$, $d_i^l < d_i^u$, and at least of one of $d_i^l$ and $d_i^u$ are finite for all $i\in\{1,\ldots,m_I\}$. The bounds in~\eqref{bounds} are such that $x^l\in\mathbb{R}^{n}\cup\{-\infty\}$, $x^u\in\mathbb{R}^{n}\cup\{+\infty\}$, and $x_i^l < x_i^u$, $i\in\{1,\ldots,n\}$. The quantities insides brackets are the Lagrange multipliers of the constraints. Whenever a bound is infinite, the corresponding multiplier is by convention zero.

The following quantities are required by \Hi:
\begin{itemize}
\item[D1] objective and constraint functions $f(x)$, $c(x)$, $d(x)$;
\item[D2] the first-order derivatives of the above: $\nabla f(x)$, $Jc(x)$, $Jd(x)$;
\item[D3] The Hessian of the Lagrangian
  \begin{align}
\nabla^2 L(x)& = \lambda_0 \nabla^2 f(x) + \sum_{i=1}^{m_E} \lambda_i^E \nabla^2 c_i(x) + \sum_{i=1}^{m_I} \lambda_i^I \nabla^2 d_i(x).\label{spHess}
\end{align}
\item[D4] the simple bounds $x_l$ and $x_u$, the inequalities bounds: $d_l$ and $d_u$, and the right-hand size of the equality constraints $c_E$.
\end{itemize}

\subsubsection{C++ interface to solve sparse NLPs}
The above optimization problem~\eqref{spobj}-\eqref{spbounds} can be specified by using the C++ interface, namely by deriving and providing an implementation for the \texttt{hiop::hiopInterfaceSparse} abstract class.

 We present next the methods of this abstract class that needs to be implemented in order to specify the parts D1-D4 required to solve a sparse NLP  problem. 

\warningcp{Note:} All the functions that have a \texttt{bool} return type should return \texttt{false} when an error occurs, otherwise should return \texttt{true}.

\warningcp{Note:} \texttt{hiop::hiopInterfaceSparse} runs only in non-distributed/non-MPI mode. Intraprocess acceleration can be obtained using OpenMP or CUDA.

\subsubsection{Specifying the optimization problem}

All the methods of this section are ``pure'' virtual in \texttt{hiop::hiopInterfaceSparse} abstract class  and need to be provided by the user implementation.

\begin{lstlisting} 
bool get_prob_sizes(long long& n, long long& m);
\end{lstlisting} 
\noindent Provides the number of decision variables and the number of constraints ($m=m_E+m_I$).


\begin{lstlisting} 
bool get_vars_info(const long long& n, double *xlow, double* xupp, 
                   NonlinearityType* type);
\end{lstlisting} 

\noindent Provides the lower and upper bounds $x_l$ and $x_u$ on the decision variables. When a variable (let us say the $i^{th}$) has no lower or/and upper bounds, the  $i^{th}$ entry of xlow and/or xupp should be less than $-1e20$ or/and larger than $1e20$, respectively. The last argument is not used and can set to any value of the enum \texttt{hiop::hiopInterface::NonlinearityType}.

\begin{lstlisting} 
bool get_cons_info(const long long& m, double* clow, double* cupp, 
                   NonlinearityType* type);
\end{lstlisting}
\noindent Similar to the above, but for the inequality bounds $d_l$ and $d_u$. For equalities, set the corresponding entries in clow and cupp equal to the desired value (from $c_E$).


\begin{lstlisting} 
bool eval_f(const long long& n, 
            const double* x, bool new_x, 
            double& obj_value);
\end{lstlisting} 

\noindent Implement this method to compute the function value $f(x)$ in \texttt{obj\_value} for the provided decision variables $x$. The input argument \texttt{new\_x} specifies whether the variables $x$ have been changed since the previous call of one of the \texttt{eval\_} methods. Use this argument to ``buffer'' the objective and gradients function and derivative evaluations when this is possible.

\begin{lstlisting} 
bool eval_grad_f(const long long& n, 
                 const double* x, bool new_x, 
                 double* gradf);
\end{lstlisting} 

\noindent Same as above but for $\nabla f(x)$.

%%%%%%%%%%%%%%%%%%%%%%%%%%%%%%%%%%%%%%%%%%%%%%
\begin{lstlisting} 
 bool eval_cons(const long long& n, const long long& m, 
                const long long& num_cons,
				    const long long* idx_cons, const double* x, 
			       bool new_x, double* cons);
\end{lstlisting} 

\noindent Implement this method to provide the value of the constraints $c(x)$ and/or $d(x)$. The input parameter \texttt{num\_cons} specifies how many constraints (out of \texttt{m}) needs to evaluated; \texttt{idx\_cons} array specifies the indexes, which are zero-based, of the constraints  and is of size \texttt{num\_cons}. These values should be provided in \texttt{cons}, which is also an array of size \texttt{num\_cons}.
%%%%%%%%%%%%%%%%%%%%%%%%%%%%%%%%%%%%%%%%%%%%%%



\begin{lstlisting} 
bool 
eval_Jac_cons(const long long& n, const long long& m, 
			     const long long& num_cons, const long long* idx_cons,  
			     const double* x, bool new_x,
			     const int& nnzJacS, int* iJacS, int* jJacS, double* MJacS);
\end{lstlisting} 

\noindent Implement this method to provide the Jacobian of a subset of the  constraints $c(x)$ and/or $d(x)$ in \texttt{Jac}; this subset is specified by the array \texttt{idx\_cons}. The last three arguments should be used to specify the Jacobian information in sparse triplet format.  \texttt{iJacS} and \texttt{jJacS} needs to be jointly sorted: by indexes in \texttt{iJacS} and, for equal (row) indexes in \texttt{iJacS}, by indexes in \texttt{jJacS}.

Notes for implementer of this method: 
\begin{itemize}
\item[2.] When \texttt{iJacS} and \texttt{jJacS} are non-null, the implementer should provide the $(i,j)$   indexes in these arrays. 
\item[3.] When \texttt{MJacS} is non-null, the implementer should provide the values corresponding to    entries specified by \texttt{iJacS} and \texttt{jJacS}.
\item[4.] \texttt{iJacS} and \texttt{jJacS} are both either non-null or null during the same call.
\item[5.] The pair (\texttt{iJacS}, \texttt{jJacS}) and \texttt{MJacS} can be both non-null during the same call or only one of them  non-null; but they will not be both null.
\end{itemize}
%%%%%%%%%%%%%%%%%%%%%%%%%%%%%%%%%%%%%%%%%%%%%%

%%%%%%%%%%%%%%%%%%%%%%%%%%%%%%%%%%%%%%%%%%%%%%
\begin{lstlisting}
bool eval_Jac_cons(const long long& n, const long long& m, 
			          const double* x, bool new_x,
			          const long long& nsparse, const long long& ndense, 
			          const int& nnzJacS, int* iJacS, int* jJacS, double* MJacS);
\end{lstlisting} 

\noindent Evaluates the Jacobian of equality and inequality constraints \textit{in one call}. 
   
\warningcp{Note:}   HiOp will call this method whenever the implementer/user returns \texttt{false} from the previous, ``two-calls'' \texttt{eval\_Jac\_cons}. We remark that the two-calls method should return  \texttt{false} during both calls (for equalities and inequalities) made to it by \Hi in order to let \Hi know that the Jacobian should be evaluated using the one-call callback listed above.
   
   
   
The main difference from the above \texttt{eval\_Jac\_cons} is that the implementer/user of this 
    method does not have to split the constraints into equalities and inequalities; instead,
    HiOp does this internally.
   
 
   Parameters:
    \begin{itemize}
    \item  first four: number of variables, number of constraints, (primal) variables at which the
    Jacobian should be evaluated, and boolean flag indicating whether the variables \texttt{x} have
    changed since a previous call to any of the function and derivative evaluations.
     \item   \texttt{nnzJacS}, \texttt{iJacS}, \texttt{jJacS}, \texttt{MJacS}: number of nonzeros, $(i,j)$ indexes, and nonzero values of 
   the sparse Jacobian matrix.  \texttt{iJacS} and \texttt{jJacS} needs to be jointly sorted: by indexes in \texttt{iJacS} and, for equal (row) indexes in \texttt{iJacS}, by indexes in \texttt{jJacS}.
    \end{itemize}
   
\warningcp{Note:} Notes 1-5 from the previous, two-call \texttt{eval\_Jac\_cons} applies here as well.

%%%%%%%%%%%%%%%%%%%%%%%%%%%%%%%%%%%%%%%%%%%%%%


%%%%%%%%%%%%%%%%%%%%%%%%%%%%%%%%%%%%%%%%%%%%%%
\begin{lstlisting} 
bool eval_Hess_Lagr(const long long& n, const long long& m, 
			      const double* x, bool new_x, const double& obj_factor,
			      const double* lambda, bool new_lambda,
			      const long long& nsparse, const long long& ndense, 
			      const int& nnzHSS, int* iHSS, int* jHSS, double* MHSS)
\end{lstlisting} 

\noindent Evaluates the Hessian of the Lagrangian function as a sparse matrix in triplet format. 

 
 
\warningcp{Note:} Notes 1-5 from \texttt{eval\_Jac\_cons} apply to arrays \texttt{iHSS}, \texttt{jHSS}, and \texttt{MHSS} that stores the sparse part of the Hessian.

\warningcp{Note:} The array \texttt{lambda} contains first the multipliers of the equality constraints followed by the multipliers of the inequalities.

%%%%%%%%%%%%%%%%%%%%%%%%%%%%%%%%%%%%%%%%%%%%%%

\subsubsection{Calling \Hi for a \texttt{hiopInterfaceSparse} formulation}
Once the sparse NLP is coded,  the  user code needs to create a \Hi problem formulation that encapsulate the NLP representation, instantiate an optimization algorithm class, and start the numerical optimization process. Assuming that the NLP representation is implemented in a class named \texttt{NlpEx6} (that derives from \texttt{hiop::hiopInterfaceSparse}), the aforementioned sequence of steps can be performed by:
\begin{lstlisting}
#include "NlpEx6.hpp"               //the NLP representation class
#include "hiopInterface.hpp"   //HiOP encapsulation of the NLP
#include "hiopAlgFilterIPM.hpp"     //solver class
using namespace hiop;
...
NlpEx6 nlp_interface();                     //instantiate your NLP representation class
hiopNlpDenseConstraints nlp(nlp_interface); //create HiOP encapsulation
nlp.options.SetNumericValue("mu0", 0.01);  //set a non-default initial value for  barrier parameter
hiopAlgFilterIPM solver(&nlp);             //create a solver object
hiopSolveStatus status = solver.run();     //numerical optimization
double obj_value = solver.getObjective();  //get objective
...
\end{lstlisting}
Various output quantities of the numerical optimization phase (\textit{e.g.}, the optimal objective value and (primal) solution, status of the numerical optimization process, and solve statistics) can be retrieved from \Hi's \texttt{hiopAlgFilterIPM} solver object. Most commonly used such methods are: 
\begin{lstlisting}
double getObjective() const;
void getSolution(double* x) const;
hiopSolveStatus getSolveStatus() const;
int getNumIterations() const;
\end{lstlisting} 
The standalone drivers \texttt{nlpSparse\_ex6} and \texttt{nlpSparse\_ex7} inside directory \texttt{src/Drivers/} under the \Hi's root directory contain more detailed examples of the use of the sparse NLP interface of \Hi.
